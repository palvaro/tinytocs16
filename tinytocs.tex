\documentclass{tinytocs}
%\usepackage[letterpaper,scale=0.7]{geometry}

\usepackage[sorting=ydnt, backend=bibtex,maxnames=99]{biblatex}

\usepackage{url}


\newcommand\Colorhref[3][cyan]{\href{#2}{\normalsize\color{#1}#3}}

\addbibresource{tinytocs}

\title{All faults are equal but some are more equal than others: prioritization in lineage-driven fault injection}

\author{Peter Alvaro\\
UC Santa Cruz\\
\and
Kamala Ramasubramanian\\
UC Santa Cruz\\
\and
Wang-Chiew Tan\\
UC Santa Cruz\\
}

\date{}


\begin{document}

\maketitle

\section{Abstract}

Fault tolerance is redundancy.  Prior work on lineage-driven fault injection leverages data lineage to collect 
evidence of redundancy from program traces,
lead to a natural decision problem: could some set of faults in the execution have prevented a known good system outcome?   
It constructs the problem by converting the lineage (a fine-grained explanation of the sequence of computational steps 
that produced the outcome) into a SAT formula whose satisfying models represent such fault sets, and ultimately exploring
their consequences via fault injection.  
This approach has proven successful finding bugs in both low-level protocols and large-scale systems\cite{molly}.

However, the space of possible faults is often enormous, and the SAT formulation does not prioritize the space of fault combinations (in which not all fault sets are equally likely). Moreover, bounds on execution length and number of failures must be given a priori to construct the formula, resulting in a large parameter space that must be swept.

The paper describes how the problem of finding the smallest set of faults that could invalidate support for a good outcome can be posed instead as an optimization problem, whose solutions both determine the next fault set that should be explored and provide an intuitive measure of the fault-tolerance of a system.

\section{Body}

\emph{Finding the most likely set of faults that can prevent a good distributed system outcome is a minimum hitting set problem over data lineage.}


\printbibliography
\end{document}
